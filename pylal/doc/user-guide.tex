%
% =============================================================================
%
%                                   Premable
%
% =============================================================================
%

\documentclass{book}
\usepackage{fullpage}
\usepackage{pylal}


% Add your name to the author list!  (alphabetical order?)

\title{pyLAL Applications User Guide and Reference}
\author{Kipp Cannon, Thomas Cokelaer, Alexander Dietz, Steve Fairhurst}
\date{\today}


%
% =============================================================================
%
%                                   Document
%
% =============================================================================
%

\begin{document}
\maketitle
\tableofcontents
\listoftables
\listoffigures
\chapter{Coding Conventions}
\section{General}
The executables should contain some standard code at the beginning of the document such
as
\begin{verbatim}
__version__ = "$Revision$"
__date__ = "$Date$"[7:-2]
__id__ = "$Id$"
__name__="plotthinca"
\end{verbatim}
so that the version, and Id can be used within the document if needed.

\section{Naming conventions}
The output file should be 
\begin{verbatim}
<executable name>_<usertag>_<figure name>-<GPS start time>-<duration>.png
<executable name>_<usertag>-<GPS start time>-<duration>.html
<executable name>_<usertag>-<GPS start time>-<duration>.cache
\end{verbatim}
Where, executable name is defined by the variable \prog{\_\_name\_\_} (see above)


\section{arguments}
The authors should try to use the following convention for the optional
argument as much as possible
\begin{description}
\item[\progarg{--cache-file}] to be used to provide a cache file
\item[\progarg{--figure-name}] a tag to be used for naming the output figures
(see Naming conventions).
\item[\progarg{--verbose}] Be verbose
\end{description}

\section{TODO}
sort the list of arguments by alphabetical order.



\chapter{Burst Tools}
\section{\prog{ligolw\_binjfind}}

Blah blah blah.

\chapter{Inspiral Tools}
\include{plotnumtemplates}

\end{document}
